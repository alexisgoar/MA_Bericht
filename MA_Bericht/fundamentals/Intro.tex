The basic idea behind radar measurement systems is the use of electromagnetic waves to search for objects. Broadly, a modulated waveform is generated and transmitted through directive antennas. The waveform travels through space at a constant velocity (the speed of light), and reflects of any object (target) present in the measurement space. The reflected waveform is the measured and processed in order to obtain information about the present objects. This information usually includes position (range) and radial speed (Doppler), which is where the radar (RAdio Detection and Ranging) obtained its name from. 

This chapter introduces basic concepts of the signal processing used on the waveform received by the radar. The reader is referred to \cite{mahafza_radar_2002} for a more detailed explanation of each of the concepts presented here. This chapter is organized as follows. The basic terminology of radars is presented in \cref{bs:definitions}, followed by a simple signal model presented in  \cref{bs:signal}, which is used to explain the signal processing steps used to estimate the range (\cref{bs:ranging}) and the radial-velocity (\cref{bs:dopplerRanging}). In order to obtain the position of the targets in Cartesian coordinates, the measurement of the angle between the target and the boresight of the radar is required. Several algorithms to calculate this angle are presented in \cref{bs:azimuth}. Finally, the theory behind thresholding is introduced in \cref{bs:detection}. 

