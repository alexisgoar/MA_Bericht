For the following analyses a \Ac{fmcw} signal model is taken into consideration. As the name indicates, the \ac{fmcw} radar transmits a signal continuously, from which the frequency is increased with time. For \Ac{mimo} radars, as explained before, the antennas do not transmit continuously given that they are switched after each cycle. For a linear increase of the frequency, the \Ac{fmcw}-chirp signal transmitted by each of the TX elements can be modeled as 
\begin{equation}
	s_T(t) = \exp[\mj(2\pi f_ct+\pi kt^2)]\\,
\end{equation}
for $-T_c/2 \leq t \leq T_c/2$, where $T_c$ is the chirp duration, $f_c$ the carrier frequency and the chirp rate $k$ is defined by
\begin{equation}
	k = \pm B/T_c\\.
\end{equation}

Under far-field conditions, the return echo caused by an object located ar rang $R$ and an angle of $\theta$ with respect to the boresight, i.e. $(x,y) = (R\cos\theta,R\sin\theta)$, can be simplified to
\begin{equation}
	\Delta t_{ij} = \frac{2R}{c} + \frac{2x_{ij}\sin\theta}{c}\\,
\end{equation}
for the virtual element obtained from the $i$-th TX element and $j$-th RX element. The received signal results in 
\begin{equation}
	s_R(t) = As_t(t-\Delta t_{ij})\\,
\end{equation}
where the dependency of $s_R(t) = s_R^{ij}(t) $ on $i$ and $j$ has been omitted for clear notation. 

Once the signal is received, it is down-converted in a pre-processing step by multiplying it with a replica of the transmitted signal. The outcome of the multiplication results in a component at \ac{if} frequency, and an undesired component at a higher frequency. After low-pass filtering, the processed signal is
\begin{equation}
	u_{ij}(t) = s_R^{*}s_T(t) = A\exp[\mj(2\pi k \Delta t_{ij}t-\pi k\Delta  t_{ij}^2 + 2\pi f_c \Delta t_{ij})]\\.
\end{equation}

The sampled form of the \ac{if} signal given $N$ samples can be expressed as
\begin{equation}
	u_{ij}[n] = A\exp[\mj(\phi+2\pi\Psi_R n)]\\,
\end{equation}
where $\Psi_R = k\Delta t_{ij}\frac{T_c}{N} $ is the normalized frequency containing the range information and $\phi = -\pi k\Delta  t_{ij}^2 + 2\pi f_c \Delta t_{ij} $ contains the phase information. This expression can be generalized to \ac{mimo} radar for the $k$-th observation interval by
\begin{equation}\label{eq:signal}
	u_{k}[n,n_C,n_A] = \sum_{m=0}^{M}A_m\exp[\mj2\pi(\phi_m+\Psi_{R,m}n+\Psi_{D,m}n_C+\Psi_{\theta,m}n_A)] + w[n,n_C,n_A]\\,
\end{equation}
where $\Psi_D = \frac{2T_cf_0}{c_0}v_R$ and $\Psi_\theta =\sin(\theta)/2 $ are the normalized frequencies containing the information about range-rate and azimuth. Moreover, $w[n,n_C,n_A]$ is the present additive white Gaussian measurement noise. The parameters $n$, $n_C$ and $n_A$ represent which sample, chirp, and antenna is being taken into account. Please note that $\phi_m$ does not correspond to $\phi$.

The information collected from \cref{eq:signal} is usually arranged into a three-dimensional data cube as presented in \cref{fig:datacube}. Each of the dimensions correspond to a certain parameter that has to be estimated. The first dimension, contains the $N$ samples of a pulse. This information corresponds to the range estimation (\cref{bs:ranging}). The second dimension, stores the signal received by the each of the virtual elements for the same pulse. A variety of algorithms exists to process the data in this dimension and obtain and estimate for the azimuth profile (\Cref{bs:azimuth}). Finally, the third dimension, contains multiple pulses. In other words, for a single virtual channel and range bin, the third dimension contains the received signal each time the cycle of pulses being transmitted restarts, i.e. each $T_cN_t$ seconds. 

\begin{figure}[h]
	\centering
	\begin{tikzpicture}
% Reference Grid

   \tikzstyle{virt} = [draw, shape=rectangle, minimum height=0.1cm, minimum width=0.1cm, node distance=0.0005cm and 0.0005cm, line width=1pt, fill = black]
   
\draw[gray,very thin,step = 0.5] (0,0) grid (0.5,4); 
\draw[gray,very thin,step = 0.5] (1.99,0) grid (6,4); 
\draw[gray,very thin,step = 0.5] (7.49,0) grid (11.5,4); 
%\draw[gray,very thin,step = 0.5] (9.49,2) grid (13.5,6); 

\foreach \x in {7.5,8,...,11.5}
    \draw[gray,very thin] (\x,4)--($(\x,4)+(2,2)$); 
    
\foreach \x in {0,0.5,...,4}
    \draw[gray,very thin] (11.5,\x)--($(11.5,\x)+(2,2)$); 
    
\foreach \x in {0.1,0.2,...,1}
    \draw[gray,very thin] ($(11.5,4)!\x!(13.5,6)$)-- ($(11.5,0)!\x!(13.5,2)$); 
    
\foreach \x in {0.1,0.2,...,1}
\draw[gray,very thin] ($(11.5,4)!\x!(13.5,6)$)-- ($(7.5,4)!\x!(9.5,6)$);     


\node at (0,2) [anchor = south,rotate = 90] { Time-Samples}; 
\draw[->] (-0.25,4)--(-0.25,3.5); 
\draw[->] (-0.25,0.5)--(-0.25,0); 
\node at (4,0) [anchor = north] {Receive Channels};
\draw[->] (2,-0.25) --(2.4,-0.25); 
\draw[->] (5.5,-0.25)--(6,-0.25); 
\node at (12.5,1) [anchor = north, rotate = 45, align = center] {Pulse\\ Number}; 
\draw[->] (12,0)--(12.4,0.4); 
\draw[->] (13.3,1.3)--(13.7,1.7); 

\foreach \x in {2.25,2.75,...,5.75}{
       \node at (\x,5) [virt](){}; 
            \draw[->,thick] (\x,4.75)-- (\x,4); 
}       
\foreach \x in {0.25,0.75,...,3.75}
      \draw[->,thick] (0.75,\x)-- (1.75,\x); 
\node at (4,5.25)[anchor = south]{Virtual Elements}; 

\draw[->,thick](6.25,2)--(7.25,2); 
\end{tikzpicture}
	\caption{Data Cube}
	\label{fig:datacube}
\end{figure} 

In order to estimate the spectrum of the received signal and so to determine the unknown parameters $R$, $v_R$ and $\theta$ for a given target, usually a three-dimensional \ac{fft} is applied to de \Ac{if} signal
\begin{multline}
\hat{P}_k(\Psi_R,\Psi_D\Psi_\theta) = \sum_{n}\sum_{n_C}\sum_{n_A} a_R[n] a_D[n_C] a_\theta[n_A]u_k[n,n_A,n_C]\\
\times\exp(-\mj2\pi\Psi_Rn)\exp(-\mj2\pi\Psi_Dn_C)\exp(-\mj2\pi\Psi_\theta n_A) \rlap{,}
\end{multline}
\begin{multline}
\hat{P}_k(\Psi_R,\Psi_D) = \sum_{n}\sum_{n_C}\sum_{n_A} a_R[n] a_D[n_C] u_k[n,n_A,n_C]\\
\times\exp(-\mj2\pi\Psi_Rn)\exp(-\mj2\pi\Psi_Dn_C) \rlap{,}
\end{multline}
however, this usually results in a bad estimation for the azimuth range. Specifics of the signal processing will be described in the following sections. 