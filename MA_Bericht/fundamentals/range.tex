The estimation of the range and the range rate can be done by using a two-dimensional fft in the first and third dimension of the data cube.

\begin{equation}
\hat{P}_k(\Psi_R,\Psi_D) = \sum_{n}\sum_{n_C} a_R[n] a_D[n_C] u_k[n,n_A,n_C]\times\exp(-\mj2\pi\Psi_Rn)\exp(-\mj2\pi\Psi_Dn_C) \rlap{,}
\end{equation}

That's all for now, but what about Space- Adaptive Time Processing? WORK IN PROGRESS

\begin{figure}[h]
	\centering
	\begin{tikzpicture}
% Reference Grid

   \tikzstyle{virt} = [draw, shape=rectangle, minimum height=0.1cm, minimum width=0.1cm, node distance=0.0005cm and 0.0005cm, line width=1pt, fill = black]
   
\draw[gray,very thin,step = 0.5] (0,0) grid (0.5,4); 
\draw[gray,very thin,step = 0.5] (1.99,0) grid (6,4); 
\draw[gray,very thin,step = 0.5] (7.49,0) grid (11.5,4); 
%\draw[gray,very thin,step = 0.5] (9.49,2) grid (13.5,6); 

\foreach \x in {7.5,8,...,11.5}
    \draw[gray,very thin] (\x,4)--($(\x,4)+(2,2)$); 
    
\foreach \x in {0,0.5,...,4}
    \draw[gray,very thin] (11.5,\x)--($(11.5,\x)+(2,2)$); 
    
\foreach \x in {0.1,0.2,...,1}
    \draw[gray,very thin] ($(11.5,4)!\x!(13.5,6)$)-- ($(11.5,0)!\x!(13.5,2)$); 
    
\foreach \x in {0.1,0.2,...,1}
\draw[gray,very thin] ($(11.5,4)!\x!(13.5,6)$)-- ($(7.5,4)!\x!(9.5,6)$);     


\node at (0,2) [anchor = south,rotate = 90] { Time-Samples}; 
\draw[->] (-0.25,4)--(-0.25,3.5); 
\draw[->] (-0.25,0.5)--(-0.25,0); 
\node at (4,0) [anchor = north] {Receive Channels};
\draw[->] (2,-0.25) --(2.4,-0.25); 
\draw[->] (5.5,-0.25)--(6,-0.25); 
\node at (12.5,1) [anchor = north, rotate = 45, align = center] {Pulse\\ Number}; 
\draw[->] (12,0)--(12.4,0.4); 
\draw[->] (13.3,1.3)--(13.7,1.7); 

\foreach \x in {2.25,2.75,...,5.75}{
       \node at (\x,5) [virt](){}; 
            \draw[->,thick] (\x,4.75)-- (\x,4); 
}       
\foreach \x in {0.25,0.75,...,3.75}
      \draw[->,thick] (0.75,\x)-- (1.75,\x); 
\node at (4,5.25)[anchor = south]{Virtual Elements}; 

\draw[->,thick](6.25,2)--(7.25,2); 
\end{tikzpicture}
	\caption{CHANGE TO ILLUSTRATE DATA PROCESSING }
	\label{fig:datacube2}
\end{figure} 


