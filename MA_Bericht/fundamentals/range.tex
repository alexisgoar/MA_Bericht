The estimation of the range and the range rate can be done by using a two-dimensional fft in the first and third dimension of the data cube.

\begin{equation}
\hat{P}_k(\Psi_R,\Psi_D) = \sum_{n}\sum_{n_C} a_R[n] a_D[n_C] u_k[n,n_A,n_C]\times\exp(-\mj2\pi\Psi_Rn)\exp(-\mj2\pi\Psi_Dn_C) \rlap{,}
\end{equation}

That's all for now, but what about Space- Adaptive Time Processing? WORK IN PROGRESS

\begin{figure}[h]
	\centering
	\begin{tikzpicture}
% Reference Grid

   \tikzstyle{virt} = [draw, shape=rectangle, minimum height=0.1cm, minimum width=0.1cm, node distance=0.0005cm and 0.0005cm, line width=1pt, fill = black]
   
\draw[gray,very thin,step = 0.5] (7.49,0) grid (11.5,4); 
%\draw[gray,very thin,step = 0.5] (9.49,2) grid (13.5,6); 

\foreach \x in {7.5,8,...,11.5}
    \draw[gray,very thin] (\x,4)--($(\x,4)+(2,2)$); 
    
\foreach \x in {0,0.5,...,4}
    \draw[gray,very thin] (11.5,\x)--($(11.5,\x)+(2,2)$); 
    
\foreach \x in {0.1,0.2,...,1}
    \draw[gray,very thin] ($(11.5,4)!\x!(13.5,6)$)-- ($(11.5,0)!\x!(13.5,2)$); 
    
\foreach \x in {0.1,0.2,...,1}
\draw[gray,very thin] ($(11.5,4)!\x!(13.5,6)$)-- ($(7.5,4)!\x!(9.5,6)$);     


\node at (7.5,2) [anchor = south,rotate = 90] {Sample}; 
\node at (12.5,1) [anchor = north, rotate = 45, align = center] {Pulse}; 
\node at (7.5,0) [anchor = south,rotate = 90]{$1$}; 
\node at (7.5,4) [anchor = south, rotate=90] {$N$}; 
\node at (11.5,0) [anchor = north, rotate = 0]{1}; 
\node at (13.7,1.8) [anchor = north]{$N_C$}; 

\draw[->] (4,3)-- (4,4)node[anchor=south,rotate = 90] {\footnotesize Fast-Time}; 
\draw[->] (4,3)-- (4.7,3.7)node[anchor=north,rotate = 45] {\footnotesize Slow-Time}; 
\draw[->] (6,0.2)--(6,3.8) ; 
\node at (6,2)[anchor = south,rotate =90](){FFT}; 
\draw[->] (13.5,0.2)--(15.5,2.2); 
\node at (14.5,1.2)[anchor = north,rotate =45](){FFT}; 
\end{tikzpicture}
	\caption{CHANGE TO ILLUSTRATE DATA PROCESSING }
	\label{fig:datacube2}
\end{figure} 


