A \ac{mimo} setup for this research has been chosen. The concepts behind the \ac{mimo} radar will be presented briefly in this section. 

As the name implies, a \ac{mimo} radar consists of multiple transmit (TX) antennas and receive (RX) antennas. Given $M_t$ transmit and $M_r$ receive elements in an array, $M_t\times M_r$ propagation channels are obtained. This, with only $M_t
M_r$ antenna elements. Furthermore, a method to define the diversity of the TX channels is required. This can be achieved by employing time division multiplexing, frequency division multiplexing, spatial coding, and orthogonal waveforms \cite{huang_fmcw_2011}. 

Each of the  $M_t\times M_r$ propagation channels are modeled together as a virtually form array. Each element in the virtual array is placed at 
\begin{equation}
	\vec{x_{ij}} = (\vec{x}_i^{Tx}+\vec{x}_j^{Rx})/2\\,
\end{equation}
where $\vec{x}_i^{Tx}$ is the position of the $i$-TX element and $\vec{x}_j^{Rx}$ of the $j$-RX element, as shown in \cref{fig:mimo_array}

\begin{figure}[h]
	\centering
	A \ac{mimo} setup for this research has been chosen. The concepts behind the \ac{mimo} radar will be presented briefly in this section. 

As the name implies, a \ac{mimo} radar consists of multiple transmit (TX) antennas and receive (RX) antennas. Given $M_t$ transmit and $M_r$ receive elements in an array, $M_t\times M_r$ propagation channels are obtained. This, with only $M_t
M_r$ antenna elements. Furthermore, a method to define the diversity of the TX channels is required. This can be achieved by employing time division multiplexing, frequency division multiplexing, spatial coding, and orthogonal waveforms \cite{huang_fmcw_2011}. 

Each of the  $M_t\times M_r$ propagation channels are modeled together as a virtually form array. Each element in the virtual array is placed at 
\begin{equation}
	\vec{x_{ij}} = (\vec{x}_i^{Tx}+\vec{x}_j^{Rx})/2\\,
\end{equation}
where $\vec{x}_i^{Tx}$ is the position of the $i$-TX element and $\vec{x}_j^{Rx}$ of the $j$-RX element, as shown in \cref{fig:mimo_array}

\begin{figure}[h]
	\centering
	A \ac{mimo} setup for this research has been chosen. The concepts behind the \ac{mimo} radar will be presented briefly in this section. 

As the name implies, a \ac{mimo} radar consists of multiple transmit (TX) antennas and receive (RX) antennas. Given $M_t$ transmit and $M_r$ receive elements in an array, $M_t\times M_r$ propagation channels are obtained. This, with only $M_t
M_r$ antenna elements. Furthermore, a method to define the diversity of the TX channels is required. This can be achieved by employing time division multiplexing, frequency division multiplexing, spatial coding, and orthogonal waveforms \cite{huang_fmcw_2011}. 

Each of the  $M_t\times M_r$ propagation channels are modeled together as a virtually form array. Each element in the virtual array is placed at 
\begin{equation}
	\vec{x_{ij}} = (\vec{x}_i^{Tx}+\vec{x}_j^{Rx})/2\\,
\end{equation}
where $\vec{x}_i^{Tx}$ is the position of the $i$-TX element and $\vec{x}_j^{Rx}$ of the $j$-RX element, as shown in \cref{fig:mimo_array}

\begin{figure}[h]
	\centering
	A \ac{mimo} setup for this research has been chosen. The concepts behind the \ac{mimo} radar will be presented briefly in this section. 

As the name implies, a \ac{mimo} radar consists of multiple transmit (TX) antennas and receive (RX) antennas. Given $M_t$ transmit and $M_r$ receive elements in an array, $M_t\times M_r$ propagation channels are obtained. This, with only $M_t
M_r$ antenna elements. Furthermore, a method to define the diversity of the TX channels is required. This can be achieved by employing time division multiplexing, frequency division multiplexing, spatial coding, and orthogonal waveforms \cite{huang_fmcw_2011}. 

Each of the  $M_t\times M_r$ propagation channels are modeled together as a virtually form array. Each element in the virtual array is placed at 
\begin{equation}
	\vec{x_{ij}} = (\vec{x}_i^{Tx}+\vec{x}_j^{Rx})/2\\,
\end{equation}
where $\vec{x}_i^{Tx}$ is the position of the $i$-TX element and $\vec{x}_j^{Rx}$ of the $j$-RX element, as shown in \cref{fig:mimo_array}

\begin{figure}[h]
	\centering
	\input{fundamentals_figs/mimo.tex}
	\caption{\ac{mimo}-array}
		\label{fig:mimo_array}
\end{figure} 

 If the far-field condition is met for a given scatterer at position $\vec{p}$, the signal propagation path from a given TX element to the scatterer, plus the reflection path back to an RX element can be approximated as
 \begin{equation}
 	P_{ij}(p) = |\vec{p}-\vec{x}_i^{Tx}| + |\vec{p}-\vec{x}_j^{Rx}| \approx 2|\vec{p}-\vec{x}_{ij}| 
 \end{equation}
	\caption{\ac{mimo}-array}
		\label{fig:mimo_array}
\end{figure} 

 If the far-field condition is met for a given scatterer at position $\vec{p}$, the signal propagation path from a given TX element to the scatterer, plus the reflection path back to an RX element can be approximated as
 \begin{equation}
 	P_{ij}(p) = |\vec{p}-\vec{x}_i^{Tx}| + |\vec{p}-\vec{x}_j^{Rx}| \approx 2|\vec{p}-\vec{x}_{ij}| 
 \end{equation}
	\caption{\ac{mimo}-array}
		\label{fig:mimo_array}
\end{figure} 

 If the far-field condition is met for a given scatterer at position $\vec{p}$, the signal propagation path from a given TX element to the scatterer, plus the reflection path back to an RX element can be approximated as
 \begin{equation}
 	P_{ij}(p) = |\vec{p}-\vec{x}_i^{Tx}| + |\vec{p}-\vec{x}_j^{Rx}| \approx 2|\vec{p}-\vec{x}_{ij}| 
 \end{equation}
	\caption{\ac{mimo}-array}
		\label{fig:mimo_array}
\end{figure} 

 If the far-field condition is met for a given scatterer at position $\vec{p}$, the signal propagation path from a given TX element to the scatterer, plus the reflection path back to an RX element can be approximated as
 \begin{equation}
 	P_{ij}(p) = |\vec{p}-\vec{x}_i^{Tx}| + |\vec{p}-\vec{x}_j^{Rx}| \approx 2|\vec{p}-\vec{x}_{ij}| 
 \end{equation}